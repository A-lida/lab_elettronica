%! Author = alida
%! Date = 22/11/2024

% Preamble
\documentclass[@SRC@/main]{subfiles}
\graphicspath{@IMAGES@}

% tutti i pacchetti usati vanno nel main

% Document
\begin{document}

\section{Apparato sperimentale e svolgimento} \label{sec:acquisizione}
  Il circuito in \textit{Fig.\!\!\! 1} è stato realizzato su una millefori, ed è stato alimentato mediante un $alimentatore\, di\, bassa\, tensione$, erogante una differenza di potenziale di +5V. 
  Le misure di tensione sono state effettuante mediante l'ausilio di un $oscilloscopio$ con risoluzione dipendente dal fondoscala utilizzato e di un $multimetro\, digitale$, avente risoluzione pari a  mV, quest'ultimo è stato impiegato anche per acquisire le misure di corrente, per le quali si ha risoluzione di 0.01mA.
  
  
  \begin{center}
    \begin{tabular}{ ||c|c|| }
        \hline
        \multicolumn{2}{||c||}{Silicio}\\
        \hline
        Oscilloscopio (mV) & Multimetro (mA) \\
        \hline
        $(680\pm 3.7 )$ & $(44.90167035\pm 0.067)$ \\
        \hline
        $(660\pm 2.47 )$ & $(19.8\pm 0.0547)$ \\
        \hline
        $(640\pm 1.57 )$ & $(19.2\pm 0.0457)$ \\
        \hline
        $(620\pm 1.02 )$ & $(18.6\pm0.0402)$\\
        \hline
        $(600\pm  0.64)$ & $(18\pm0.0364)$ \\
        \hline
        $(590\pm 0.49 )$ & $(17.7\pm0.0349)$ \\
        \hline
        $(580\pm 0.42 )$ & $(17.4\pm0.0342)$ \\
        \hline
        $(570\pm 0.33 )$ & $(17.1\pm0.0333)$ \\
        \hline
        $(560\pm 0.26 )$ & $(16.8\pm0.0326)$ \\
        \hline
        $(540\pm 0.17 )$ & $(16.2\pm0.0317)$ \\
        \hline
        $(520\pm 0.12  )$ & $(15.6\pm0.0312)$ \\
        \hline 
        $(500\pm 0.08 )$ & $( 15\pm0.0308)$ \\
        \hline 
        $(480\pm 0.05 )$ & $(14.4\pm0.0305 )$ \\
        \hline 
        $(460\pm 0.04 )$ & $( 13.8\pm0.0304)$ \\
        \hline 
        $(420\pm 0.02 )$ & $(12.6 \pm0.0302)$ \\
        \hline 
        
        
    \end{tabular}   
    \captionof{table}{Misura della caratteristica del Silicio tramite \textit{multimetro digitale} ed \textit{oscilloscopio}, con relative incertezze.} 
\end{center}

\begin{center}
  \begin{tabular}{ ||c|c|| }
  \hline
  \multicolumn{2}{||c||}{Germanio}\\
  \hline
  Oscilloscopio (mV) & Multimetro (mA) \\
  \hline
  $(360\pm 5.11 )$ & $(22.72971623\pm 0.0811)$ \\
  \hline
  $(340\pm 3.71 )$ & $(22.45083517\pm 0.0671)$ \\
  \hline
  $(320\pm 2.77 )$ & $(22.1846794\pm 0.0577)$ \\
  \hline
  $(300\pm 2.2 )$ & $(13.45362405\pm 0.052)$\\
  \hline
   $(290\pm 1.87)$ & $(13.25481045\pm0.0487)$ \\
  \hline
   $(280\pm 1.57 )$ & $(13.05986217\pm0.0457)$ \\
  \hline
   $(270\pm 1.33 )$ & $(12.86895489\pm0.0433)$ \\
  \hline
   $(260\pm 1.1 )$ & $(12.68227109\pm0.041)$ \\
  \hline
  $(250\pm 0.93 )$ & $(12.5\pm0.0393)$ \\
  \hline
   $(240\pm 0.8 )$ & $(12.32233744\pm0.038)$ \\
  \hline
  $(230\pm 0.68  )$ & $(12.14948559\pm0.0368)$ \\
  \hline 
  $(220\pm 0.56 )$ & $(11.98165264\pm0.0356)$ \\
  \hline 
  $(210\pm 0.47 )$ & $(11.81905242\pm0. 0347)$ \\
  \hline 
  $(200\pm 0.38 )$ & $(11.66190379\pm0.0338)$ \\
  \hline 
  $(190\pm 0.31 )$ & $(11.51043005\pm0.0331)$ \\
  \hline 
  $(180\pm 0.26 )$ & $(11.36485812 \pm0.0326)$ \\
  \hline 
  $(170\pm 0.2 )$ & $(11.22541759 \pm0.032)$ \\
  \hline 
  $(160\pm 0.17 )$ & $(11.0923397	 \pm0.0317)$ \\
  \hline 
  $(150\pm 0.14 )$ & $(10.9658561 \pm0.0314)$ \\
  \hline 
  $(140\pm 0.11 )$ & $(10.84619749 \pm0.0311)$ \\
  \hline 
  $(130\pm 0.09 )$ & $(10.73359213\pm0.0309)$ \\
  \hline 
  $(120\pm 0.07 )$ & $(5.381449619 \pm0.0307)$ \\
  \hline 
  $(116\pm 0.06 )$ & $(5.301924179 \pm0.0306)$ \\
  \hline 
  $(112\pm 0.05 )$ & $(5.22394487 \pm0.0305)$ \\
  \hline 
 
  \end{tabular}   
  \captionof{table}{Misura della caratteristica del Germanio tramite \textit{multimetro digitale} ed \textit{oscilloscopio}, con relative incertezze.} 
 \end{center}

\end{document}