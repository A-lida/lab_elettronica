%! Author = paolo
%! Date = 18/11/2024

% Preamble
\documentclass[11pt]{article}

% Pacchetti per cose che non sono simboli
\usepackage{graphicx} % per mettere i prefissi alle path delle immagini nei subfiles
\usepackage{circuitikz}
\usepackage{caption}
\usepackage{subcaption}
\usepackage[margin=2cm]{geometry}
\usepackage{microtype} % Sinceramente non ho capito a cosa serva ma dicono sia magico

% Paccheti per simboli
\usepackage{amsmath}
\usepackage{textcomp}
\usepackage{listings}

% Pacchetto per includere altri file (da includere per ultimo)
\usepackage{subfiles}

% Setup dei pacchetti
\graphicspath{{@IMAGES@}}

\renewcommand{\tablename}{Tabella}
\renewcommand{\figurename}{Figura}

% Document
\begin{document}

\title{\textbf{Misura della caratteristica di due diodi a giunzione p-n}}
\author{Castagnoli Alida, Forni Paolo}
\date{8 novembre 2024}
%DA AGGIUNGERE IL TURNO (CREDO 2) E IL NUMERO DEL TAVOLO (NO IDEA) !!!!!!!!!!!!!!!!!!!!!!!
\maketitle


\vspace{-23pt}  % Riduce lo spazio sopra l'abstract

\begin{abstract}
  L'esperimento ha la finalità di misurare la curva caratteristica \mbox{I-V} di due diodi a semiconduttore,
  al Germanio (Ge) e al Silicio (Si).
  Successivamente, mediante due fit distinti, si vuole stimare la corrente inversa $I_{0}$ e il prodotto del
  fattore di idealità con l'equivalente in Volt della temperatura $\eta V_{T}$.

  \noindent Si riportano i valori misurati, con le relative incertezze associate:
  \vspace{-1pt}
  \begin{align*}
    &\textnormal{Silicio:} \;
    &I_{0} &= (2.3 \pm 4.1)\cdot 10^{-9} \;\;\textnormal{A} &
    \eta V_{T} &= (47.8 \pm 7.3)\;\;\textnormal{mV} & \\
    &\textnormal{Germanio:} \;
    &I_{0} &= (9.1 \pm 2.2) \cdot 10^{-6} \;\;\textnormal{A} &
    \eta V_{T} &= (54.4 \pm 3.1)\;\;\textnormal{mV} &
  \end{align*}

\end{abstract}
\subfile{@SRC@/sections/introduzione.tex}
\subfile{@SRC@/sections/acquisizione.tex}
\newpage
\subfile{@SRC@/sections/analisi.tex}
\subfile{@SRC@/sections/conclusioni.tex}
%\newpage
\appendix

\subfile{@SRC@/appendix/propagazione_errori_log.tex}

\end{document}
