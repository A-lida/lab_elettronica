%! Author = alida
%! Date = 18/12/2024

% Preamble
\documentclass[../main.tex]{subfiles}
\graphicspath{{../images}}
% tutti i pacchetti usati vanno nel main
% Document
\begin{document}

    \section{Apparato sperimentale e svolgimento} \label{sec:acquisizione}

    \subsection{Apparato sperimentale}\label{subsec:apparato-sperimentale}

    \subsection{Acquisizione dati}\label{subsec:acquisizione-dati}

    \begin{table}[ht]
        \centering
        \subfile{../tabelle_e_circuito/100uA.tex}
        \captionsetup{justification=centering} % Opzionale, migliora l'allineamento
        \caption{Misura della caratteristica del diodo al Silicio mediante \textit{multimetro digitale} e
        \textit{oscilloscopio}, con i fondo scala utilizzati. Si riportano anche le incertezze associate,
            il cui calcolo è consultabile in METTERE REF APPENDICE.}
        \label{tab:100uA}

    \end{table}

    \begin{table}[ht]
        \centering
        \subfile{../tabelle_e_circuito/200uA.tex}
        \captionsetup{justification=centering} % Opzionale, migliora l'allineamento
        \caption{Misura della caratteristica del diodo al Silicio mediante \textit{multimetro digitale} e
        \textit{oscilloscopio}, con i fondo scala utilizzati. Si riportano anche le incertezze associate,
            il cui calcolo è consultabile in METTERE REF APPENDICE.}
        \label{tab:200uA}

    \end{table}

\end{document}
