%! Author = alida
%! Date = 22/11/2024

% Preamble
\documentclass[@MAIN@]{subfiles}
%\graphicspath{@IMAGES@}
% tutti i pacchetti usati vanno nel main
% Document
\begin{document}

    \section{Apparato sperimentale e svolgimento} \label{sec:acquisizione}

    \subsection{Apparato sperimentale}\label{subsec:apparato-sperimentale}
    L'apparato sperimentale si compone di due circuiti, realizzati su una millefori
    e alimentati mediante un \textit{alimentatore di bassa tensione},
    erogante una differenza di potenziale di +5V costante.

    \begin{figure}[ht]
        \centering
        \begin{subfigure}[b]{0.4\textwidth}
            \centering
            \subfile{../tabelle_e_circuito/circuito_calibrazione}
            \caption{Calibrazione}
            \label{fig:circuito-calibrazione}
        \end{subfigure}
        \hfill
        \begin{subfigure}[b]{0.55\textwidth}
            \centering
            \subfile{../tabelle_e_circuito/circuito_acquisizione}
            \caption{Caratteristica}
            \label{fig:circuito-caratteristica}
        \end{subfigure}
        \caption{Circuiti realizzati: (a) calibrazione, (b) caratteristica.}
        \label{fig:figura_circuiti}
    \end{figure}

    \newpage
    Entrambi i circuiti, presentati schematicamente in Figura~\ref{fig:figura_circuiti}, si compongono di
    un \textit{potenziometro} di portata pari a 1k\textohm, di un amperometro e
    di un voltmetro.
    Il circuito in Figura~\ref{fig:circuito-caratteristica} presenta anche il diodo
    a semiconduttore.

    Le misure di tensione sono state effettuante mediante l'ausilio di un
    \textit{oscilloscopio analogico} e di un \textit{multimetro digitale}, avente
    risoluzione pari a 0.1mV, quest'ultimo è stato impiegato anche per le misure
    di corrente, per le quali si ha risoluzione di 0.01mA.
    \vspace{0.5pt}

    \subsection{Acquisizione dati}\label{subsec:acquisizione-dati}

    Al fine di ottenere risultati attendibili è stata condotta una presa dati per valutare la
    calibrazione degli strumenti di misura, realizzando il circuito in Figura~\ref{fig:circuito-calibrazione}. Successivamente,
    mediante la realizzazione del circuito schematizzato in Figura~\ref{fig:circuito-caratteristica}
    sono stati acquisiti i dati di corrente e tensione per ricostruire la caratteristica dei diodi.\newline

    \noindent In seguito si riportano i risultati delle misure effettuate:\newline
    \vspace{1.5pt}
%
%    \begin{table}[ht]
%        \centering
%        \begin{subtable}[t]{.45\textwidth}
%            \centering
%            \subfile{../tabelle_e_circuito/silicio}
%            \captionsetup{justification=centering} % Opzionale, migliora l'allineamento
%            \caption{Silicio} % Sottocaption
%            \label{tab:silicio}
%        \end{subtable}
%        \hfill
%        \begin{subtable}[t]{.45\textwidth}
%            \centering
%            \subfile{../tabelle_e_circuito/germanio}
%            \captionsetup{justification=centering} % Opzionale
%            \caption{Germanio} % Sottocaption
%            \label{tab:germanio}
%        \end{subtable}
%
%        \vspace{0.5pt} % Spazio opzionale tra minipage e caption generale
%
%        \caption{Misura della caratteristica dei semiconduttori mediante \textit{multimetro digitale} e
%        \textit{oscilloscopio}, con incertezze: (a) silicio, (b) germanio.}
%        \label{tab:tabelle_semiconduttori}
%
%    \end{table}
    \begin{table}[ht]
        \centering
        \subfile{../tabelle_e_circuito/germanio}
        \captionsetup{justification=centering} % Opzionale
        \caption{Misura della caratteristica del diodo al Germanio mediante \textit{multimetro digitale} e
        \textit{oscilloscopio}, con i fondo scala utilizzati. Si riportano anche le incertezze associate,
            il cui calcolo è consultabile in METTERE REF APPENDICE.}
        \label{tab:germanio}
    \end{table}

    \begin{table}[ht]
        \centering
        \subfile{../tabelle_e_circuito/silicio}
        \captionsetup{justification=centering} % Opzionale, migliora l'allineamento
        \caption{Misura della caratteristica del diodo al Silicio mediante \textit{multimetro digitale} e
        \textit{oscilloscopio}, con i fondo scala utilizzati. Si riportano anche le incertezze associate,
            il cui calcolo è consultabile in METTERE REF APPENDICE.}
        \label{tab:silicio}

    \end{table}
    \begin{table}[ht]
        \centering
        \subfile{../tabelle_e_circuito/calibrazione}
        \caption{Misura di tensione effettuate mediante \textit{multimetro digitale} ed \textit{oscilloscopio},
            con relative incertezze. Si riportano inoltre i fondo scala utilizzati.}
        \label{tab:calibrazione}

    \end{table}

\end{document}
