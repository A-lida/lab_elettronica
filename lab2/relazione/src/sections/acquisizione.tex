%! Author = alida
%! Date = 18/12/2024

% Preamble
\documentclass[../main.tex]{subfiles}
\graphicspath{{../images}}
% tutti i pacchetti usati vanno nel main
% Document
\begin{document}

    \section{Apparato sperimentale e svolgimento} \label{sec:acquisizione}

    \subsection{Apparato sperimentale}\label{subsec:apparato-sperimentale}

    L'apparato sperimentale si compone di un \textit{transistor BJT} composto da un
    semiconduttore al silicio avente doppia giunzione di tipo p-n-p, e da due
    \textit{potenziometri} aventi resistenze pari a 1~k\textohm\;e 100~k\textohm. Al fine di acquisire
    le misure di corrente e tensione sono stati adoperati rispettivamente un \textit{multimetro digitale},
    avente risoluzione pari a 0.01~mA e un \textit{oscilloscopio analogico}.
    Il circuito realizzato è presentato schematicamente in Figura\ref{fig:circuito}.

    \begin{figure}[h!]
        \centering
        \subfile{../tabelle_e_circuiti/circuito_acquisizione}
        \caption{Circuito realizzato per l'acquisizione della caratteristica di uscita del BJT}
        \label{fig:circuito}
    \end{figure}
    \newpage
    \subsection{Acquisizione dati}\label{subsec:acquisizione-dati}

    La prima operazione che si effettuata, preliminare alla realizzazione
    del circuito, è quella di settare una resistenza da 50~k\textohm\;
    sul potenziometro da 100~k\textohm, in modo da non danneggiare il
    transistor nella fase di inserimento dello stesso nel circuito. \\
    Successivamente si collega l'emettitore a massa, il collettore al piedino centrale
    del potenziometro da 1~k\textohm, mentre la base viene collegata al multimetro
    in modo da poter settare la corrente in ingresso. Si collega quindi
    l'alimentazione, erogando una differenza di potenziale costante di -5~V e
    si imposta, mediante l'ausilio del multimetro, una corrente di base
    a -100~\textmu A. Tale multimetro viene collegato al piedino
    centrale del potenziometro da 1~k\textohm\; e al collettore, a quest'ultimo
    viene collegato anche un capo dell'oscilloscopio, il cui altro terminale è collegato a terra.
    Si ottiene quindi la configurazione del circuito presentato in Figura\ref{fig:circuito} e si procede
    pertanto all'acquisizione della caratteristica, variando la
    resistenza del potenziometro da 1~k\textohm.
    Si ripete il precedente ragionamento settando una corrente di base
    di -200~\textmu A, i risultati delle acquisizioni sono tabulati in seguito:

    \begin{table}[ht]
        \centering
        \subfile{../tabelle_e_circuiti/100uA}
        \captionsetup{justification=centering} % Opzionale, migliora l'allineamento
        \caption{Misura della caratteristica del diodo al Silicio mediante \textit{multimetro digitale} e
        \textit{oscilloscopio}, con i fondo scala utilizzati. Si riportano anche le incertezze associate,
            il cui calcolo è consultabile in METTERE REF APPENDICE.}
        \label{tab:100uA}

    \end{table}

    \begin{table}[ht]
        \centering
        \subfile{../tabelle_e_circuiti/200uA}
        \captionsetup{justification=centering} % Opzionale, migliora l'allineamento
        \caption{Misura della caratteristica in uscita del transistor BJT \textit{multimetro digitale} e
        \textit{oscilloscopio}, con i fondo scala utilizzati. Si riportano anche le incertezze associate,
            il cui calcolo è consultabile in \ref{...}.}
        \label{tab:200uA}

    \end{table}

\end{document}
