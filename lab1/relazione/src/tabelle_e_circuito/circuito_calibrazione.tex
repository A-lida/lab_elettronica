\documentclass[@MAIN@]{subfiles}

\begin{document}
\hspace*{\fill}
\begin{minipage}[b]{0.45\textwidth}
  \centering
  \begin{circuitikz}
    \draw (5, 1) node[oscopeshape, rotate=-90, rotated instruments](O){};
    \draw (3.5, 1) node[smetershape, rotate=-90, rotated instruments, t=V](V){};
    \draw
    (0,2) to[battery1]
      (0,0) -- 
      (1.5,0) to[american potentiometer]
      (1.5,2) --
      (0,2)
    (1.5,2) -- 
      (3,2) --
      (3,2 |- V.in 1) to[short, -*]
      (V.in 1)
    (V.in 2) to[short, *-]
      (V.in 2 -| 3,0) --
      (3,0) --
      (1.5, 0)
    (3,2) --
      (4.5,2) --
      (4.5,2 |- O.in 1) to[short, -*]
      (O.in 1)
    (O.in 2) to[short, *-]
      (O.in 2 -| 4.5, 0) --
      (4.5, 0) --
      (3,0)
    (0,0) node[ground] {}
    ;
    \draw [line width=0.85pt]
    (0.955,1) --
      (0.955,0)
    ;
  \end{circuitikz}
%  \captionof{figure}{Schema del circuito realizzato per l'acquisizione dei dati}
%  \label{fig:circuito}
\end{minipage}
\hspace*{\fill}
\end{document}











