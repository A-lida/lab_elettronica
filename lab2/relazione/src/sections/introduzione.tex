%! Author = alida
%! Date = 18/12/2024

% Preamble
\documentclass[../main.tex]{subfiles}
\usepackage{hyperref}
\usepackage{hyperref}
\graphicspath{{../images}}

% tutti i pacchetti usati vanno nel main

% Document
\begin{document}
    \section{Introduzione} \label{sec:introduzione}


    L'esperimento prende in esame il comportamento di un transistor a giunzione bipolare,
    in breve BJT, utilizzato nella configurazione a emettitore comune. Esso è un componente
    elettronico costituito da tre strati di materiale drogato,
    un semiconduttore al silicio, in cui lo strato centrale ha drogaggio opposto agli
    altri due, in modo da formare una doppia giunzione p-n; il transistor
    utilizzato nell'esperimento presenta una giunzione di tipo p-n-p. Ad ogni strato è associato
    un terminale: quello centrale prende il nome di base,
    quelli esterni sono detti collettore ed emettitore. La configurazione ad emettitore
    comune definisce la base del transistor come terminale di ingresso, mentre il
    collettore assume il ruolo di terminale di uscita. \newline

    Il fine dell'esperimento è quello di misurare la caratteristica in uscita del BJT,
    ovvero l'andamento della corrente di collettore ($I_C$) in funzione della tensione
    tra collettore ed emettitore ($V_{CE}$) per due differenti correnti di base ($I_B$). Si vuole
    quindi calcolare la tensione di early $V_A$ e la conduttanza $g$, infine si
    vuole calcolare il guadagno di corrente del transistor,$\beta$, per un valore fissato di $V_{CE}$
    nella regione attiva.


%    poi fissiamo un valore
%    di tensione e facciamo il rapporto
%    delle differenze di corrente tra le due caratteristiche
%    e otteniamo il guadagnoi del transistor
%    Vogliamo un'alimentazione di -5V
%    Il transistor BJT verrà utilizzato nella configurazione a
%    emettitore comune, quindi definiamo la base come terminale
%    di ingresso e il collettore come terminale di uscita.
%
%    Utilizziamo il potenziometro da 100~k\ohm per settare la corrente di base, ovvero
%    la corrente che circola nel terminale di ingresso del transistor. Vogliamo fissare la corrente sulla base
%    quindi una volta

\end{document}