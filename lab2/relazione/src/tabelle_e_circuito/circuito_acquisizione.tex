\documentclass[../main.tex]{subfiles}
\begin{document}
%    \begin{figure}[h!]
        \centering
        \begin{circuitikz}
            \draw (1,2.7) node[smetershape, rotated instruments, t=A](A){};
            \draw (2,1) node[oscopeshape, rotate=-90, rotated instruments](O){};
            \draw
            (-1,0) to[battery1]
            (-1,3.2) --
            (0,3.2) to[american potentiometer, name=p1]
            (0,0) --
            (-1, 0)
            (-1,0) --
            (4,0) to[american potentiometer, name=p2]
            (4,3.2) --
            (0,3.2)
            (p2.wiper) node[pnp, xscale=-0.8, yscale=-0.8, tr circle, anchor=west](T) {};
            \draw
            (p1.wiper) --
            (p1.wiper -| A.in 1) to[short, -*]
            (A.in 1)
            (A.in 2) to[short, *-]
            (A.in 2 |- T.C) --
            (1.5, 2 |- T.C) --
            (1.5, 2 |- O.in 1) to[short, -*]
            (O.in 1)
            (O.in 2) to[short, *-]
            (O.in 2 -| 1.5, 0) --
            (1.5, 0)
            (T.C) --
            (1.5, 2 |- T.C)
            (T.E) --
            (-1,0 -| T.E)
            (-1,0) node[ground] {}
            ;
        \end{circuitikz}
        \caption{Circuito realizzato per l'acquisizione della caratteristica di uscita del BJT}
        \label{fig:circuito}

%      \end{figure}

\end{document}
