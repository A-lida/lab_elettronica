%! Author = paolo
%! Date = 26/12/2024

% Preamble
\documentclass[../main.tex]{subfiles}
\graphicspath{{../images}}
% tutti i pacchetti usati vanno nel main

% Document
\begin{document}

\section{Calcolo errori sul coefficiente $\beta$}
  \label{sec:errori-beta}
  Per evitare ambiguità, in questa sezione, per riferirsi ai valori
  di $\varDelta I_C$ e $\varDelta I_B$ si useranno i simboli
  $\delta I_C$ e $\delta I_B$, in modo da non confonderli con gli
  errori ad essi associati $\Delta I_C$ e $\Delta I_B$.\\

  \noindent Riprendendo la formula per $\beta$:
  \begin{equation*}
    \beta = \frac{\delta I_C}{\delta I_B}
  \end{equation*}
  È immediato vedere che il suo errore relativo sarà pari alla somma
  di quelli di $\delta I_C$ e $\delta I_B$.
  Tra i due è facile convincersi che quello associato a
  $\delta I_B$ sia il dominante una volta calcolato:
  \begin{equation*}
    \left\{
    \begin{array}{@{}l@{}}
      \Delta_r I_B = \frac{\Delta I_B}{\delta I_B} = \frac{\Delta I_{B1} + \Delta I_{B2}}{I_{B_2} - I_{B_1}} \\[1ex]
      \Delta I_{Bi} = 0.3 \text{ (Si veda l'appendice~\ref{sec:propagazione-errori-misure})}
    \end{array}\right. \implies
    \Delta_r I_B = \frac{0.06}{0.1} = 0.06 \cdot 10 = 60\%
  \end{equation*}
  Con un errore errore del 60\% già solo su $\delta I_B$ è
  lapalissiano che la misura non possa essere precisa.

\end{document}