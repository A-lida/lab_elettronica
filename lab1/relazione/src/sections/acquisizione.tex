%! Author = alida
%! Date = 22/11/2024

% Preamble
\documentclass[@MAIN@]{subfiles}
\graphicspath{@IMAGES@}

% tutti i pacchetti usati vanno nel main

% Document
\begin{document}

\section{Apparato sperimentale e svolgimento} \label{sec:acquisizione}
  Il circuito in Figura~\ref{fig:circuito} è stato realizzato su una millefori, ed è stato alimentato mediante un \textit{alimentatore di bassa tensione}, erogante una differenza di potenziale di +5V. 
  Le misure di tensione sono state effettuante mediante l'ausilio di un \textit{oscilloscopio} e di un \textit{multimetro digitale}, avente risoluzione pari a $0.1$mV, quest'ultimo è stato
  impiegato anche per acquisire le misure di corrente, per le quali si ha risoluzione di 0.01mA.
  \\ \\
  \subfile{../tabelle_e_circuito/circuito}
  \\
  \begin{center}
    \subfile{../tabelle_e_circuito/calibrazione}
  \end{center}

  \begin{minipage}{.45\textwidth}
    \subfile{../tabelle_e_circuito/silicio}
  \end{minipage}
  \hfill
  \begin{minipage}{.45\textwidth}
    \subfile{../tabelle_e_circuito/germanio}
  \end{minipage}

\end{document}
