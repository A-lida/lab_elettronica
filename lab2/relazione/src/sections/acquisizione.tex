%! Author = alida
%! Date = 18/12/2024

% Preamble
\documentclass[../main.tex]{subfiles}
\graphicspath{{../images}}
% tutti i pacchetti usati vanno nel main
% Document
\begin{document}

    \section{Apparato sperimentale e svolgimento} \label{sec:acquisizione}

    \subsection{Apparato sperimentale}\label{subsec:apparato-sperimentale}

    L'apparato sperimentale si compone di un \textit{transistor BJT} composto da un
    semiconduttore al silicio avente doppia giunzione di tipo p-n-p, e da due
    \textit{potenziometri} aventi resistenze pari a 1~\;k\textohm\;e 100~\;k\textohm. Al fine di acquisire
    le misure di corrente e tensione sono stati adoperati rispettivamente un \textit{multimetro digitale},
    avente risoluzione pari a 0.01~\;mA e un \textit{oscilloscopio analogico}.
    \newpage

    \subsection{Acquisizione dati}\label{subsec:acquisizione-dati}
    \vspace{0.2cm}
    \subsection*{Settaggio corrente di base}

    Le caratteristiche che si intendono acquisire durante l'esperimento, corrispondo alle correnti
    di base di -100\;\textmu A e -200\;\textmu A. È quindi necessario impostare tali valori prima di realizzare
    il circuito di acquisizione della caratteristica.

    \begin{figure}[h!]
        \centering
        \subfile{../tabelle_e_circuiti/circuito_corrente_base}
        \caption{Circuito realizzato per settare la corrente di base $I_B$ entrante nel transistor BJT.}
        \label{fig:circuito-corrente-base}
    \end{figure}

    \noindent La prima operazione effettuata, è stata quella di settare una resistenza da 50~\;k\textohm\;
    sul potenziometro da 100~\;k\textohm, in modo da non danneggiare il
    transistor nella fase di inserimento dello stesso nel circuito. \\
    Successivamente si è collegato l'emettitore a massa, il collettore al piedino centrale
    del potenziometro da 1~\;k\textohm, mentre la base è stata collegata al multimetro
    in modo da poter settare la corrente in ingresso. In seguito, si alimenta il circuito erogando
    una differenza di potenziale costante di -5~V e si imposta, mediante l'ausilio del multimetro,
    una corrente di base pari a -100~\;\textmu A. Una rappresentazione schematica del circuito realizzato
    è riportata in Figura~\ref{fig:circuito-corrente-base}.

    \subsection*{Misura della caratteristica in uscita}
    Settata la corrente di base, si procede collegando il i capi del multimetro al piedino
    centrale del potenziometro da 1~\;k\textohm\; e al collettore, a quest'ultimo
    viene collegato anche un capo dell'oscilloscopio, il cui altro terminale è collegato a terra.
    Si ottiene quindi la configurazione del circuito presentato in Figura~\ref{fig:circuito-caratteristica}
    e si procede pertanto all'acquisizione della caratteristica, variando la
    resistenza del potenziometro da 1~\;k\textohm.
    Al fine di ottenere misure corrette della caratteristica in uscita, è stato necessario adottare
    un metodo sistematico durante l'acquisizione: sono stati misurati due valori di corrente a distanza di 10 s
    l'uno dall'altro, per la singola misura di tensione. Questa operazione si è rivelata necessaria in quanto
    per valori di corrente dell'ordine del ..... la dipendenza della ..... dalla temperatura non è trascurabile.
%    TODO: finire questa frase decentemente!

    \begin{figure}[ht]
        \centering
        \subfile{../tabelle_e_circuiti/circuito_acquisizione}
        \caption{Circuito per l'acquisizione della caratteristica in uscita del transistor BJT.}
        \label{fig:circuito-caratteristica}
    \end{figure}

    Si ripete l'acquisizione descritta in precedenza, settando una corrente di base
    di -200~\textmu A, i risultati delle acquisizioni sono tabulati in seguito:
    \clearpage
    \begin{table}[!ht]
        \centering
        \subfile{../tabelle_e_circuiti/100uA}
        \captionsetup{justification=centering} % Opzionale, migliora l'allineamento
        \caption{Misura della caratteristica in uscita del transistor BJT, per una corrente di base pari a -100\;\textmu A. Il
        valore di corrente riportato rappresenta la media dei due valori misurati. Si
        riportano anche i fondo scala utilizzati e le incertezze associate, il cui calcolo
        è consultabile in ...} %    TODO: mettere ref quando ci sarà
        \label{tab:100uA}

    \end{table}
    \begin{table}[!ht]
        \centering
        \subfile{../tabelle_e_circuiti/200uA}
        \captionsetup{justification=centering} % Opzionale, migliora l'allineamento
        \caption{Misura della caratteristica in uscita del transistor BJT, per una corrente di base pari a -200\;\textmu A. Il
        valore di corrente riportato rappresenta la media dei due valori misurati. Si
        riportano anche i fondo scala utilizzati e le incertezze associate, il cui calcolo
        è consultabile in ...} %    TODO: mettere ref quando ci sarà
        \label{tab:200uA}

    \end{table}

\end{document}
