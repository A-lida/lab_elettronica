\documentclass[@MAIN@]{subfiles}

\begin{document}
\hspace*{\fill}
\begin{minipage}[b]{0.45\textwidth}
  \centering
  \begin{circuitikz}
    \draw (4.5,1) node[smetershape, rotate=-90, rotated instruments, t=A](A){};
    \draw (3,3.5) node[oscopeshape](O){};
    \draw
    (2,2) to[american potentiometer]
      (0,2) to[battery1]
      (0,0) 
    (2,2) to[full diode] 
      (4,2) --
      (4,2 |- A.in 1) to[short, -*]
      (A.in 1)
    (A.in 2) to[short, *-]
      (A.in 2 -| 4,0) --
      (4,0) --
      (0,0)
    (2,2) --
      (2,3) --
      (2,3 -| O.in 1) to[short, -*]
      (O.in 1)
    (O.in 2) to[short, *-]
      (O.in 2 |- 4,3) --
      (4,3) --
      (4,2)
      (0,0) node[ground] {}
    ;
    \draw [line width=0.85pt]
    (1,1.5) --
      (1,0)
    ;
  \end{circuitikz}
%  \captionof{figure}{Schema del circuito realizzato per l'acquisizione dei dati}
%  \label{fig:circuito}
\end{minipage}
\hspace*{\fill}
\end{document}
