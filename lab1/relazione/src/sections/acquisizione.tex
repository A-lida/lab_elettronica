%! Author = alida
%! Date = 22/11/2024

% Preamble
\documentclass[@MAIN@]{subfiles}
\usepackage{textcomp}
\graphicspath{@IMAGES@}

% tutti i pacchetti usati vanno nel main

% Document
\begin{document}

    \section{Apparato sperimentale e svolgimento} \label{sec:acquisizione}

    \subsection{Apparato sperimentale}\label{subsec:apparato-sperimentale}
    L'apparato sperimentale si compone di due circuiti, realizzati su una millefori
    e alimentati mediante un \textit{alimentatore di bassa tensione},
    erogante una differenza di potenziale di +5V costante.

    \begin{figure}[ht]
        \centering
        \begin{subfigure}[b]{0.4\textwidth}
            \centering
            \subfile{../tabelle_e_circuito/circuito_calibrazione}
            \caption{Calibrazione}
            \label{fig:circuito-calibrazione}
        \end{subfigure}
        \hfill
        \begin{subfigure}[b]{0.55\textwidth}
            \centering
            \subfile{../tabelle_e_circuito/circuito_acquisizione}
            \caption{Caratteristica}
            \label{fig:circuito-caratteristica}
        \end{subfigure}
%        \vspace{0.5em} % Spazio opzionale tra minipage e caption generale
        \caption{Circuiti realizzati: (a) calibrazione, (b) caratteristica.}
        \label{fig:figura_circuiti}
    \end{figure}

    \newpage
    Entrambi i circuiti, presentati schematicamente in Figura~1, si compongono di
    un \textit{potenziometro} di portata pari a 1k\textohm, di un amperometro e
    di un voltmetro.
    Il circuito in Figura~\ref{fig:circuito-caratteristica} presenta anche il diodo
    a semiconduttore.

    Le misure di tensione sono state effettuante mediante l'ausilio di un
    \textit{oscilloscopio analogico} e di un \textit{multimetro digitale}, avente
    risoluzione pari a 0.1mV, quest'ultimo è stato impiegato anche per le misure
    di corrente, per le quali si ha risoluzione di 0.01mA.

    \subsection{Acquisizione dati}\label{subsec:acquisizione-dati}

    Al fine di ottenere risultati attendibili, sono state effettuate due tipologie
    di misure: inizialmente è stata condotta una presa dati per valutare la
    calibrazione degli strumenti di misura, successivamente sono stati acquisiti
    i dati di corrente e tensione per ricostruire la caratteristica dei diodi. \\

    Il primo circuito .....
    in seguito si riportano le misure
    \\ \\
    \begin{center}
        \subfile{../tabelle_e_circuito/calibrazione}
    \end{center}


    \begin{table}[ht]
        \centering
        \begin{minipage}[t]{.45\textwidth}
            \centering
            \subfile{../tabelle_e_circuito/silicio}
            \captionsetup{justification=centering} % Opzionale, migliora l'allineamento
            \caption*{(a) Silicio} % Sottocaption
        \end{minipage}
        \hfill
        {\begin{minipage}[t]{.45\textwidth}
             \centering
             \subfile{../tabelle_e_circuito/germanio}
             \captionsetup{justification=centering} % Opzionale
             \caption*{(b) Germanio} % Sottocaption
        \end{minipage}}

        \vspace{0.5em} % Spazio opzionale tra minipage e caption generale

        \caption{Misura della caratteristica dei semiconduttori mediante \textit{multimetro digitale} e \textit{oscilloscopio}, con incertezze: (a) silicio, (b) germanio.}
        \label{tab:tabelle_semiconduttori}

    \end{table}
\end{document}
