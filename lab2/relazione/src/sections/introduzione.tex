%! Author = alida
%! Date = 18/12/2024

% Preamble
\documentclass[../main.tex]{subfiles}
\graphicspath{{../images}}

% tutti i pacchetti usati vanno nel main

% Document
\begin{document}
    \section{Introduzione} \label{sec:introduzione}

    L'esperimento prende in esame il comportamento di un transistor a giunzione bipolare,
    in breve BJT, utilizzato nella configurazione a emettitore comune.
    Esso è un componente elettronico costituito da tre strati di materiale drogato,
    un semiconduttore al silicio, in cui lo strato centrale ha drogaggio opposto agli
    altri due, in modo da formare una doppia giunzione p-n; il transistor
    utilizzato nell'esperimento presenta una giunzione di tipo p-n-p.
    Ad ogni strato è associato un terminale: quello centrale prende il nome di base,
    quelli esterni sono detti collettore ed emettitore.
    La configurazione ad emettitore comune definisce la base del transistor come
    terminale di ingresso, mentre il collettore assume il ruolo di terminale di uscita.\\

    Secondo le equazioni di Ebers-Moll:
    \begin{equation}
        I_C = \beta I_B
        \label{eq:ebers-moll}
    \end{equation}
    la corrente di collettore $I_C$ di BJT ideale nella regione
    attiva non dipende dalla tensione tra collettore ed
    emettitore $V_{CE}$ ma solo dal guadagno di corrente $\beta$ e
    dalla corrente che fluisce nella base $I_B$.
    Tuttavia in un transistor reale la larghezza della depletion
    region aumenta al crescere di $V_{CE}$, ciò fa diminuire la
    larghezza di base e di conseguenza aumentare il $\beta$.
    Questo fenomeno è noto come \textit{Effetto Early} e trasforma
    l'Eq.~\eqref{eq:ebers-moll} nella relazione fenomenologica:
    \begin{equation}
        I_C(V_{CE}) = \beta I_B \left( 1 + \frac{V_{CE}}{V_A} \right)
        \label{eq:early}
    \end{equation}
    La quale descrive delle rette non più parallele all'asse X,
    bensì un fascio passante per il punto $(V_A, 0)$.\\

    Ai fini dell'esperimento è importante ricordare che i transistor BJT sono sensibili
    alla temperatura, che tra le altre cose comporta un aumento della
    corrente di collettore pari al 7\% per kelvin; nelle condizioni prese in esame
    questo effetto non è trascurabile e sono state quindi adottate delle contromisure per
    ridurne l'impatto.\\

    Il fine dell'esperimento è quello di misurare la caratteristica in uscita del BJT,
    ovvero l'andamento della corrente di collettore ($I_C$) in funzione della tensione
    tra collettore ed emettitore ($V_{CE}$) per due differenti correnti di base ($I_B$). \\
    Si vuole quindi calcolare la tensione di early $V_A$, la resistenza $b$ e la
    conduttanza $g$ definite come:
    \begin{equation}
        b^{-1} = g = \frac{\varDelta I_C}{\varDelta V_{CE}}
        \label{eq:g/b}
    \end{equation}
    Infine si vuole calcolare il guadagno di corrente del transistor
    $\beta$, per un valore fissato di $V_{CE}$ nella regione attiva.

%    TODO: un caratteristica per ogni corrente di base scelta, effetto joule fa shiftare la caratteristica


\end{document}