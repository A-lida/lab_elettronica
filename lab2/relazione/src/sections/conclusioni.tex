%! Author = alida
%! Date = 18/12/2024

% Preamble
\documentclass[../main.tex]{subfiles}
\graphicspath{{../images}}

% tutti i pacchetti usati vanno nel main

% Document
\begin{document}

    \section{Conclusioni} \label{sec:conclusioni}
    I risultati evidenziano un'incompatibilità tra le tensioni di Early misurate
    per le due correnti di base, questo fatto è in disaccordo con le previsioni teoriche.\\

    I valori di ${\chi^{2}_{\nu}}$ ottenuti dai fit indicano una sovrastima delle
    incertezze associate alle misure, dovuta alla risoluzione dell'oscilloscopio, ma anche
    un accordo con la funzione di fit.\\

    Si ipotizza quindi che la discrepanza tra i
    valori della tensione di Early ottenuti, possa essere dovuta alle variazioni di
    temperatura causate dall'effetto Joule. Questa ipotesi è corroborata dalla presenza di andamenti
    non lineari della corrente di collettore nelle regioni di tensione più alta. In particolare,
    nella caratteristica a -200~\textmu A questo effetto è più evidente a causa delle
    maggiori correnti in gioco.\\

    La misura del guadagno di corrente risulta essere compatibile con la previsione teorica (\textasciitilde 100),
    tuttavia presenta un errore relativo del 64\%, risultando pertanto inattendibile.\\

    \noindent Si riportano in seguito i valori misurati, con le relative
    incertezze:
    \vspace{0.2cm}
    \begin{center}
        \begin{tabular}{llll}
            \centering
            -100 \textmu A: & $V_A = (15.2 \pm 1.1)$ V &
            $b = (0.935 \pm 0.060)$ $\text{V} \cdot \text{mA}^{-1}$ &
            $g = (1.069 \pm 0.069)$ mA$\cdot \text{V}^{-1}$ \\[0.1cm]
            -200 \textmu A: & $V_A = (12.46 \pm 0.93)$ V &
            $b = (0.453 \pm 0.029)$ $\text{V} \cdot \text{mA}^{-1}$ &
            $g = (2.21 \pm 0.14)$ mA$\cdot \text{V}^{-1}$
        \end{tabular} \\[0.3cm]
        $\beta(V_{CE} = -2 \text{ V}) = 136 \pm  87$
    \end{center}


\end{document}