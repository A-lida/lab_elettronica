%! Author = alida
%! Date = 18/12/2024

% Preamble
\documentclass[11pt]{article}

% Pacchetti per cose che non sono simboli
\usepackage{graphicx} % per mettere i prefissi alle path delle immagini nei subfiles
\usepackage{circuitikz}
\usepackage{caption}
\usepackage{subcaption}
\usepackage[margin=2cm]{geometry}
\usepackage{microtype} % Sinceramente non ho capito a cosa serva ma dicono sia magico

% Paccheti per simboli
\usepackage{amsmath}
\usepackage{textcomp}
\usepackage{listings}

% Pacchetto per includere altri file (da includere per ultimo)
\usepackage{subfiles}

% Setup dei pacchetti
\graphicspath{{images}}

\renewcommand{\tablename}{Tabella}
\renewcommand{\figurename}{Figura}

% Document
\begin{document}

% \LaTeX

    \title{\textbf{
        Misura della caratteristica di uscita di un trasistor BJT P-N-P in configurazione
        a emettitore comune
    }}
    \author{Castagnoli Alida, Forni Paolo}
    \date{22 novembre 2024}
%DA AGGIUNGERE IL TURNO (CREDO 2) E IL NUMERO DEL TAVOLO (NO IDEA) !!!!!!!!!!!!!!!!!!!!!!!
    \maketitle


    \vspace{-23pt}  % Riduce lo spazio sopra l'abstract

    \begin{abstract}
        L'esperimento vuole misurare le caratteristica in uscita
        di un transistor BJT p-n-p, al silicio, in configurazione a
        emettitore comune, fissando due valori di corrente di base,
        pari a -100~\textmu A e -200~\textmu A.
        Successivamente, mediante un fit lineare sui dati acquisiti
        si vogliono stimare la tensione di Early $V_A$, la
        resistenza di uscita $b$ e la conduttanza in uscita $g$,
        inoltre si calcola guadagno di corrente $\beta$ tra le due
        caratteristiche in corrispondenza di una tensione ($V_{CE}$)
        di 2 V. \\
        
        \noindent Si riportano in seguito i valori misurati, con le relative
        incertezze:
        \vspace{0.2cm}
        \begin{center}
            \begin{tabular}{llll}
                \centering
                -100 \textmu A: & $V_A = (15.2 \pm 1.1)$ V &
                    $b = (0.935 \pm 0.060)$ $\text{V} \cdot \text{mA}^{-1}$ &
                    $g = (1.069 \pm 0.069)$ mA$\cdot \text{V}^{-1}$ \\[0.1cm]
                -200 \textmu A: & $V_A = (12.46 \pm 0.93)$ V &
                    $b = (0.453 \pm 0.029)$ $\text{V} \cdot \text{mA}^{-1}$ &
                    $g = (2.21 \pm 0.14)$ mA$\cdot \text{V}^{-1}$
            \end{tabular} \\[0.3cm]
            $\beta(V_{CE} = -2 \text{ V}) = 136 \pm  87$
        \end{center}

    \end{abstract}
    \subfile{sections/introduzione.tex}
    \subfile{sections/acquisizione.tex}
    \clearpage
    \subfile{sections/analisi.tex}
    \newpage
    \subfile{sections/conclusioni.tex}
    \newpage

    \appendix
    \section*{Appendice}
    \subfile{appendix/propagazione_errori_log.tex}

    \subfile{appendix/errori_su_beta.tex}


\end{document}
