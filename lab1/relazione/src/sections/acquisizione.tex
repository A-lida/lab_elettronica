%! Author = alida
%! Date = 22/11/2024

% Preamble
\documentclass[@SRC@/main]{subfiles}
\graphicspath{@IMAGES@}

% tutti i pacchetti usati vanno nel main

% Document
\begin{document}

\section{Apparato sperimentale e svolgimento} \label{sec:acquisizione}
  Il circuito in \textit{Fig.\!\!\! 1} è stato realizzato su una millefori, ed è stato alimentato mediante un \textit{alimentatore di bassa tensione}, erogante una differenza di potenziale di +5V. 
  Le misure di tensione sono state effettuante mediante l'ausilio di un \textit{oscilloscopio} e di un \textit{multimetro digitale}, avente risoluzione pari a $0.1$mV, quest'ultimo è stato impiegato anche per acquisire le misure di corrente, per le quali si ha risoluzione di 0.01mA.
  

  \begin{center}
    \begin{tabular}{ ||c|c|| }
        \hline
        \multicolumn{2}{||c||}{Calibrazione}\\
        \hline
        Multimetro (mV) & Oscilloscopio (mV) \\
        \hline
        $806.0\pm 3.2$ & $800\pm 47$ \\
        \hline
        $723.0\pm 3.1$ & $720\pm 45$ \\
        \hline
        $687.0\pm 3.0$ & $680\pm 45$ \\
        \hline
        $573.1\pm 1.1$ & $580\pm 27$ \\
        \hline
        $513.60\pm 0.97$ & $520\pm 25$ \\
        \hline
        $436.10\pm 0.85$ & $440\pm 24$ \\
        \hline
        $339.60\pm 0.71$ & $340\pm 22$ \\
        \hline
        $288.50\pm 0.63$ & $290\pm 13$ \\
        \hline
        $249.20\pm 0.57$ & $250\pm 13$ \\
        \hline
        $200.40\pm 0.50$ & $200\pm 12$ \\
        \hline
        $152.70\pm 0.43$ & $150\pm 11$ \\
        \hline 
        $98.60\pm 0.35$ & $100.0\pm 5.0$ \\
        \hline 
        $79.10\pm 0.32$ & $80.0\pm 4.7$ \\
        \hline 
        $48.80\pm 0.27$ & $50.0\pm 2.5$ \\
        \hline 
        $25.40\pm 0.24$ & $26.0\pm 2.1$ \\
        \hline 
        
        
    \end{tabular}   
    \captionof{table}{Misura di tensione effettuate mediante \textit{multimetro digitale} ed \textit{oscilloscopio}, con relative incertezze.} 
\end{center}

  
  \begin{center}
    \begin{tabular}{ ||c|c|| }
        \hline
        \multicolumn{2}{||c||}{Silicio}\\
        \hline
        Oscilloscopio (mV) & Multimetro (mA) \\
        \hline
        $680\pm 45$ & $3.70\pm0.067$ \\
        \hline
        $660\pm 45$ & $2.470\pm0.055$ \\
        \hline
        $640\pm 44$ & $1.570\pm0.046$ \\
        \hline
        $620\pm 44$ & $1.020\pm0.040$\\
        \hline
        $600\pm 27$ & $0.640\pm0.036$ \\
        \hline
        $590\pm 27$ & $0.490\pm0.035$ \\
        \hline
        $580\pm 27$ & $0.420\pm0.034$ \\
        \hline
        $570\pm 26$ & $0.330\pm0.033$ \\
        \hline
        $560\pm 26$ & $0.260\pm0.033$ \\
        \hline
        $540\pm 26$ & $0.170\pm0.032$ \\
        \hline
        $520\pm 25$ & $0.120\pm0.031$ \\
        \hline 
        $500\pm 25$ & $0.080\pm0.031$ \\
        \hline 
        $480\pm 25$ & $0.050\pm0.031$ \\
        \hline 
        $460\pm 24$ & $0.040\pm0.030$ \\
        \hline 
        $420\pm 24$ & $0.020\pm0.030$ \\
        \hline 
        
        
    \end{tabular}   
    \captionof{table}{Misura della caratteristica del Silicio tramite \textit{multimetro digitale} ed \textit{oscilloscopio}, con relative incertezze.} 
\end{center}

\begin{center}
  \begin{tabular}{ ||c|c|| }
  \hline
  \multicolumn{2}{||c||}{Germanio}\\
  \hline
  Oscilloscopio (mV) & Multimetro (mA) \\
  \hline
  $360\pm 23$ & $5.110\pm 0.081$ \\
  \hline
  $340\pm 22$ & $3.710\pm 0.067$ \\
  \hline
  $320\pm 22$ & $2.770\pm 0.058$ \\
  \hline
  $300\pm 13$ & $2.200\pm 0.052$\\
  \hline
  $290\pm 13$ & $1.870\pm 0.049$ \\
  \hline
  $280\pm 13$ & $1.570\pm 0.046$ \\
  \hline
  $270\pm 13$ & $1.330\pm 0.043$ \\
  \hline
  $260\pm 13$ & $1.100\pm 0.041$ \\
  \hline
  $250\pm 13$ & $0.930\pm 0.039$ \\
  \hline
  $240\pm 12$ & $0.800\pm 0.038$ \\
  \hline
  $230\pm 12$ & $0.680\pm 0.037$ \\
  \hline 
  $220\pm 12$ & $0.560\pm 0.036$ \\
  \hline 
  $210\pm 12$ & $0.470\pm 0. 035$ \\
  \hline 
  $200\pm 12$ & $0.380\pm 0.034$ \\
  \hline 
  $190\pm 12$ & $0.310\pm 0.033$ \\
  \hline 
  $180\pm 11$ & $0.260\pm 0.033$ \\
  \hline 
  $170\pm 11$ & $0.200\pm 0.032$ \\
  \hline 
  $160\pm 11$ & $0.170\pm 0.032$ \\
  \hline 
  $150\pm 11$ & $0.140\pm 0.031$ \\
  \hline 
  $140\pm 11$ & $0.110\pm 0.031$ \\
  \hline 
  $130\pm 11$ & $0.090\pm 0.031$ \\
  \hline 
  $120.0\pm 5.4$ & $0.070\pm 0.031$ \\
  \hline 
  $116.0\pm 5.3$ & $0.060\pm 0.031$ \\
  \hline 
  $112.0\pm 5.2$ & $0.050\pm 0.031$ \\
  \hline 
 
  \end{tabular}   
  \captionof{table}{Misura della caratteristica del Germanio tramite \textit{multimetro digitale} ed \textit{oscilloscopio}, con relative incertezze.} 
 \end{center}

\end{document}
