%! Author = paolo
%! Date = 06/12/2024

% Preamble
\documentclass[@SRC@/main]{subfiles}
% tutti i pacchetti usati vanno nel main

% Document
\begin{document}
    \section*{Appendice}
    \section{Stima delle incertezze sui dati sperimentali}
    \label{sec:stima-delle-incertezze-sui-dati-sperimentali}
    Nei casi in cui la sensibilità dello strumento lo permetteva, le incertezze associate sono
    state arrotondate a due cifre significative.
    \subsection*{Multimetro}
    Il multimetro digitale utilizzato è il \textit{FLUKE 175}.
    Con esso sono state effettuate sia misure di corrente che di tensione:
    \begin{itemize}
        \item \textbf{Misure di tensione}: Secondo quanto riportato dal costruttore, il multimetro
        ha una risoluzione di 0.1~mV nel range fino a 600~mV. L'incertezza da associare alla misura
        è lo $0.15\%$ della lettura più 0.2~mV;
        \item \textbf{Misure di corrente}: il costruttore riporta, nel range fino a
        60~mA, una risoluzione di 0.01~mA e un errore pari all'$1\%$ della
        lettura più 0.03~mA;
    \end{itemize}

    \subsection*{Oscilloscopio}
    \noindent Le misure di tensione sono state acquisite mediante l'ausilio di un oscilloscopio analogico,
    la formula per calcolare l'incertezza associata è:
    
        \begin{equation*}
            \sigma = \sqrt {\left( \sigma_Z + \sigma_L  \right)^2 + \sigma_C^2}
        \end{equation*}
    
\vspace{0.1cm}
    \noindent Le incertezze sulla lettura ($\sigma_L$) e sullo zero ($\sigma_Z$) sono state considerate dipendenti
    e valutate mediante la formula:
    \vspace{0.1cm}


        \begin{equation*}
            \sigma_L = \sigma_Z = \frac{fondo\textnormal{ }scala}{5} \frac{1}{2}
        \end{equation*}

    \noindent Ove si è aggiunto il fattore moltiplicativo $\frac{1}{2}$ in quanto il segnale risultava
    essere sufficientemente pulito.\newline\newline
    \noindent Infine, è stato valutato anche l'errore riportato dal costruttore ($\sigma_C$), pari
    al 3\% del valore misurato.

    \section{Propagazione degli errori in scala semi-logaritmica}
    \label{sec:propagazione-errori-log}

    Al fine di eseguire un fit lineare sui dati in scala semi-logaritmica e, successivamente,
    riottenere i valori reali dei risultati dei fit, è stato necessario propagare le incertezze
    sulle singole misure.
    Essendo il logaritmo e l'esponenziale funzioni non lineari, si è
    dovuti ricorrere alla formula generale per la propagazione degli
    errori:
    \begin{equation*}
        \Delta f(x_0) = f'(x_0) \cdot \Delta x
        \vspace{2pt}
    \end{equation*}
    Che nei casi trattati da questo esperimento diventa:
    \begin{align*}
        \vspace{4pt}
        \Delta \ln (I) &= \frac{\Delta I}{I} \\
        \Delta I_0 = \Delta e^{\ln (I_0)} &= e^{\ln (I_0)}
        \cdot \Delta \ln (I_0)
    \end{align*}
\end{document}