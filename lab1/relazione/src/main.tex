%! Author = paolo
%! Date = 18/11/2024

% Preamble
\documentclass[11pt]{article}

% Pacchetti per cose che non sono simboli
\usepackage{graphicx} % per mettere i prefissi alle path delle immagini nei subfiles
\usepackage{circuitikz}
\usepackage{caption}
\usepackage[margin=2cm]{geometry}

% Paccheti per simboli
\usepackage{amsmath}

\usepackage{listings}



% Pacchetto per includere altri file (da includere per ultimo)
\usepackage{subfiles}

% Document
\begin{document}

\LaTeX

\title{\textbf{Misura della caratteristica di due diodi a giunzione p-n}}
\author{Castagnoli Alida, Forni Paolo}
\date{8 novembre 2024}
%DA AGGIUNGERE IL TURNO (CREDO 2) E IL NUMERO DEL TAVOLO (NO IDEA) !!!!!!!!!!!!!!!!!!!!!!!
\maketitle

\begin{abstract}
    L'esperimento ha la finalità di misurare la curva caratteristica \mbox{I-V} di due diodi a semiconduttore, al silicio (Si) e al germanio (Ge). 
    Successivamente, mediante due fit distinti, si vuole stimare la corrente inversa $I_{0}$ e il prodotto del fattore di idealità con l'equivalente in Volt della temperatura $\eta V_{T}$.  
    Si riportano i valori attesi, con le relative incertezze associate, rispettivamente per il diodo al silicio e al germanio:

    \begin{align*}
        I_{0} &= (... \pm ...) \;\;\textnormal{mA}, & 
        \eta V_{T} &= (... \pm ...)\;\;\textnormal{mV}, &
    \end{align*}

    \vspace{-0.6cm}

    \begin{align*}
        I_{0} &= (... \pm ...) \;\;\textnormal{mA}, & 
        \eta V_{T} &= (... \pm ...)\;\;\textnormal{mV}, &
    \end{align*}

\end{abstract}
% esempio di come includere un file
\subfile{@SRC@/sections/example.tex}
\subfile{@SRC@/sections/introduzione.tex}
\subfile{@SRC@/sections/acquisizione.tex}




\end{document}
