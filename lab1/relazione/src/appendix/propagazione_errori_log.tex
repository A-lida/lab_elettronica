%! Author = paolo
%! Date = 06/12/2024

% Preamble
\documentclass[@SRC@/main]{subfiles}
\usepackage{amsmath}

% tutti i pacchetti usati vanno nel main

% Document
\begin{document}

\section{Propagazione errori}\label{sec:propagazione-errori}

  \subsection{Stima delle incertezze sui dati sperimentali}
\label{subsec:propagazione-errori-misure}

\subsection{Propagazione degli errori in scala semi-logaritmica}
\label{subsec:propagazione-errori-log}

    Per poter eseguire il test del $\chi^2$ sui dati anche in scala
    semi-logaritmica, e per riottenere i valori reali dei risultati
    dei fit, si sono dovute propagare le loro incertezze.
    Non essendo il logaritmo e l'esponenziale funzioni lineari, si è
    dovuti ricorrere alla formula generale per la propagazione degli
    errori:
    \begin{equation*}
      \Delta f(x_0) = f'(x_0) \cdot \Delta x
    \end{equation*}
    Che nei casi trattati da questo esperimento diventa:
    \begin{align*}
      \Delta \ln (I) &= \frac{\Delta I}{I} \\
      \Delta I_0 = \Delta e^{\ln (I_0)} &= e^{\ln (I_0)}
        \cdot \Delta \ln (I_0)
    \end{align*}
\end{document}