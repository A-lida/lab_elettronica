%! Author = paolo
%! Date = 06/12/2024

% Preamble
\documentclass[@SRC@/main]{subfiles}
% tutti i pacchetti usati vanno nel main

% Document
\begin{document}
\section{Stima delle incertezze sui dati sperimentali}
  \label{sec:stima-delle-incertezze-sui-dati-sperimentali}
  Nei casi in cui la sensibilità dello strumento lo permetteva, gli errori sono
  stati arrotondati a due cifre significative.
  \subsection*{Multimetro}
    Il multimetro digitale utilizzato è il \textit{FLUKE 175}.
    Con esso sono state effettuate sia misure di corrente che di tensione:
    \begin{itemize}
      \item \textbf{Misure di tensione}: Secondo quanto riportato dal costruttore nel
      range fino a 600~mV il multimetro ha una risoluzione di 0.1~mV e
      l'errore da associare alla misura è il $15\%$ della lettura più 0.2~mV;
      \item \textbf{Misure di corrente}: il costruttore riporta, nel range fino a
      60~mA, una risoluzione di 0.01 mA e un errore pari all'$1\%$ della
      lettura più 0.03~mA;
    \end{itemize}

  \subsection*{Oscilloscopio}
    L'oscilloscopio è stato utilizzato per delle misure di tensione.
    La formula per calcolare l'errore da associare alla misura è:
    % TODO Capire a cosa è associato quell'errore
    \begin{equation*}
      \sqrt {\left( \sigma_0 + \sigma_{non mi ricordo}  \right)^2 + \sigma_{lettura}^2}
    \end{equation*}
    Il costruttore riporta un errore sullo zero ($\sigma_0$) e un
    errore sul ... ($\sigma_{...}$) pari al 3\% del fondo scala, e un
    errore sulla lettura ($\sigma_{lettura}$) uguale al 5\% della
    lettura moltiplicata per $\frac{1}{2}$.


\section{Propagazione degli errori in scala semi-logaritmica}
  \label{sec:propagazione-errori-log}

  Per poter eseguire il test del $\chi^2$ sui dati anche in scala
  semi-logaritmica, e per riottenere i valori reali dei risultati
  dei fit, si sono dovute propagare le loro incertezze.
  Non essendo il logaritmo e l'esponenziale funzioni lineari, si è
  dovuti ricorrere alla formula generale per la propagazione degli
  errori:
  \begin{equation*}
    \Delta f(x_0) = f'(x_0) \cdot \Delta x
  \end{equation*}
  Che nei casi trattati da questo esperimento diventa:
  \begin{align*}
    \Delta \ln (I) &= \frac{\Delta I}{I} \\
    \Delta I_0 = \Delta e^{\ln (I_0)} &= e^{\ln (I_0)}
    \cdot \Delta \ln (I_0)
  \end{align*}
\end{document}