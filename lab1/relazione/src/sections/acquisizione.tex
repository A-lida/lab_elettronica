%! Author = alida
%! Date = 22/11/2024

% Preamble
\documentclass[@SRC@/main]{subfiles}
\graphicspath{@IMAGES@}

% tutti i pacchetti usati vanno nel main

% Document
\begin{document}

\section{Apparato sperimentale e svolgimento} \label{sec:acquisizione}
  Il circuito in \textit{Fig.\!\!\! 1} è stato realizzato su una millefori, ed è stato alimentato mediante un \textit{alimentatore di bassa tensione}, erogante una differenza di potenziale di +5V. 
  Le misure di tensione sono state effettuante mediante l'ausilio di un \textit{oscilloscopio} e di un \textit{multimetro digitale}, avente risoluzione pari a $0.1$mV, quest'ultimo è stato impiegato anche per acquisire le misure di corrente, per le quali si ha risoluzione di 0.01mA.
  
  
  \begin{center}
    \begin{tabular}{ ||c|c|| }
        \hline
        \multicolumn{2}{||c||}{Silicio}\\
        \hline
        Oscilloscopio (mV) & Multimetro (mA) \\
        \hline
        $680\pm 3.7 $ & $44.90167035\pm 0.067$ \\
        \hline
        $660\pm 2.47$ & $19.8\pm 0.0547$ \\
        \hline
        $640\pm 1.57$ & $19.2\pm 0.0457$ \\
        \hline
        $620\pm 1.02$ & $18.6\pm0.0402$\\
        \hline
        $600\pm  0.64$ & $18\pm0.0364$ \\
        \hline
        $590\pm 0.49 $ & $17.7\pm0.0349$ \\
        \hline
        $580\pm 0.42 $ & $17.4\pm0.0342$ \\
        \hline
        $570\pm 0.33 $ & $17.1\pm0.0333$ \\
        \hline
        $560\pm 0.26 $ & $16.8\pm0.0326$ \\
        \hline
        $540\pm 0.17 $ & $16.2\pm0.0317$ \\
        \hline
        $520\pm 0.12 $ & $15.6\pm0.0312$ \\
        \hline 
        $500\pm 0.08 $ & $ 15\pm0.0308$ \\
        \hline 
        $480\pm 0.05 $ & $14.4\pm0.0305$ \\
        \hline 
        $460\pm 0.04 $ & $ 13.8\pm0.0304$ \\
        \hline 
        $420\pm 0.02 $ & $12.6 \pm0.0302$ \\
        \hline 
        
        
    \end{tabular}   
    \captionof{table}{Misura della caratteristica del Silicio tramite \textit{multimetro digitale} ed \textit{oscilloscopio}, con relative incertezze.} 
\end{center}

\begin{center}
  \begin{tabular}{ ||c|c|| }
  \hline
  \multicolumn{2}{||c||}{Germanio}\\
  \hline
  Oscilloscopio (mV) & Multimetro (mA) \\
  \hline
  $360\pm 5.11$ & $22.72971623\pm 0.0811$ \\
  \hline
  $340\pm 3.71$ & $22.45083517\pm 0.0671$ \\
  \hline
  $320\pm 2.77$ & $22.1846794\pm 0.0577$ \\
  \hline
  $300\pm 2.2$ & $13.45362405\pm 0.052$\\
  \hline
   $290\pm 1.87$ & $13.25481045\pm 0.0487$ \\
  \hline
   $280\pm 1.57$ & $13.05986217\pm 0.0457$ \\
  \hline
   $270\pm 1.33$  & $12.86895489\pm 0.0433$ \\
  \hline
   $260\pm 1.1$ & $12.68227109\pm 0.041$ \\
  \hline
  $250\pm 0.93$ & $12.5\pm 0.0393$ \\
  \hline
   $240\pm 0.8$ & $12.32233744\pm 0.038$ \\
  \hline
  $230\pm 0.68$ & $12.14948559\pm 0.0368$ \\
  \hline 
  $220\pm 0.56$ & $11.98165264\pm 0.0356$ \\
  \hline 
  $210\pm 0.47$ & $11.81905242\pm 0. 0347$ \\
  \hline 
  $200\pm 0.38$ & $11.66190379\pm 0.0338$ \\
  \hline 
  $190\pm 0.31$ & $11.51043005\pm 0.0331$ \\
  \hline 
  $180\pm 0.26$ & $11.36485812\pm 0.0326$ \\
  \hline 
  $170\pm 0.2$ & $11.22541759\pm 0.032$ \\
  \hline 
  $160\pm 0.17 $ & $11.0923397\pm 0.0317$ \\
  \hline 
  $150\pm 0.14 $ & $10.9658561\pm 0.0314$ \\
  \hline 
  $140\pm 0.11 $ & $10.84619749\pm 0.0311$ \\
  \hline 
  $130\pm 0.09 $ & $10.73359213\pm 0.0309$ \\
  \hline 
  $120\pm 0.07 $ & $5.381449619\pm 0.0307$ \\
  \hline 
  $116\pm 0.06 $ & $5.301924179\pm 0.0306$ \\
  \hline 
  $112\pm 0.05 $ & $5.22394487\pm 0.0305$ \\
  \hline 
 
  \end{tabular}   
  \captionof{table}{Misura della caratteristica del Germanio tramite \textit{multimetro digitale} ed \textit{oscilloscopio}, con relative incertezze.} 
 \end{center}

\end{document}
