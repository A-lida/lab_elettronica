%! Author = paolo 
%! Date = 18/11/2024

% Preamble
\documentclass[@SRC@/main]{subfiles}
\graphicspath{@IMAGES@}

% tutti i pacchetti usati vanno nel main

% Document
\begin{document}

    \section{Introduzione} \label{sec:introduzione}
    L'esperimento prende in esame il comportamento di un diodo reale, un elemento circuitale
    costituito da una giunzione \textit{p-n}, in regime di polarizzazione diretta.
    L'andamento della corrente che attraversa un diodo in funzione della tensione,
    sotto queste condizioni, è descritto dalla legge:
    \begin{equation}
        \label{eq:caratteristiche}
        I(V_D) = I_0 \cdot e^{\frac{V_D}{\eta V_T}}
    \end{equation}
    L'esperimento vuole misurare la \textit{corrente inversa} $I_0$ e il valore
    $\eta V_T$, prodotto del \textit{fattore di idealità} $\eta$ e
    dell'\textit{equivalente in volt} della temperatura $V_T$. \\
    La dipendenza dalla temperatura di $I_0$ è stata trascurata vista l'entità
    delle correnti utilizzate $\sim 0.01-1$~mA.


\end{document}