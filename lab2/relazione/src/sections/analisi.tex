%! Author = alida
%! Date = 18/12/2024

% Preamble
\documentclass[../main.tex]{subfiles}
\graphicspath{{../images}}

% tutti i pacchetti usati vanno nel main

% Document
\begin{document}

    \section{Analisi}\label{sec:analisi}
%    \subfile{../../relazione/images}
        Nella fase di analisi, è stato effettuato un fit lineare ai dati
        appartenenti alla regione attiva delle caratteristiche in
        uscita, riportati nelle
        Tabelle~\ref{tab:100uA}~e~\ref{tab:200uA}.
        Per la precisione, al fine di ricavare i parametri
        desiderati direttamente dal fit, questo è stato eseguito
        secondo la funzione $y = g \cdot ( x - V_A)$ dove $g$ è la
        pendenza della retta, mentre $V_A$ rappresenta l'ascissa
        dell'intercetta con l'asse X.

    % TODO: foto grafici
    \begin{table}[ht]
        \centering
        \begin{subtable}[t]{.45\textwidth}
            \centering
            \begin{tabular}{||c|c||}
                \hline
                \multicolumn{2}{||c||}{Corrente di base di -100\;\textmu A} \\
                \hline
                \rule{0pt}{3ex} $V_A$ (\textnormal{A}) & $\varDelta V_{CE} / \varDelta I_C$ (\textohm) \\[1ex]
                \hline
                $(...\pm...)\cdot $    & $...\pm...$                                   \\
                \hline
            \end{tabular}
            \caption{-100}
            \label{tab:fit-100uA}
        \end{subtable}
        \hfill
        \begin{subtable}[t]{.45\textwidth}
            \centering
            \begin{tabular}{||c|c||}

                \hline
                \multicolumn{2}{||c||}{Corrente di base di -200\;\textmu A} \\
                \hline

                \rule{0pt}{3ex} $V_A$ (\textnormal{V}) & $\frac{\varDelta V_{CE}}{\varDelta I_C}$ (\textohm) \\ [1ex]
                \hline
                $...\pm... $           & $...\pm...$                                          \\
                \hline
            \end{tabular}
            \caption{-200\;\textmu A.}
            \label{tab:fit-200uA}
        \end{subtable}

        \vspace{0.5pt} % Spazio opzionale tra minipage e caption generale

        \caption{Risultati dei fit lineari effettuati sulle misure delle caratteristiche in uscita del transistor BJT,
            corrispondenti a una corrente di base di (a) -100\;\textmu A, (b) -200\;\textmu A.}
        \label{tab:fit_caratteristiche}

    \end{table}
    % TODO: capire quanle tabella ti piace di più
%    \begin{table}[ht]
%        \centering
%        \begin{tabular}{||c|c||}
%            \hline
%            \multicolumn{2}{||c||}{Corrente di base di -100\;\textmu A} \\
%            \hline
%            $V_A$ (\textnormal{A}) & $\varDelta V_{CE} / \varDelta I_C$ (\textohm) \\
%            \hline
%            $(...\pm...)\cdot $    & $...\pm...$                                   \\
%            \hline
%        \end{tabular}
%        \caption{Risultati dei fit lineari effettuati sulle misure delle caratteristiche in uscita del transistor BJT,
%            corrispondente a una corrente di base di -100\;\textmu A.}
%        \label{tab:fit-100uA}
%    \end{table}

%    \begin{table}[ht]
%        \centering
%        \begin{tabular}{||c|c||}
%            \hline
%            \multicolumn{2}{||c||}{Corrente di base di -200\;\textmu A} \\
%            \hline
%            $V_A$ (\textnormal{V}) & $\eta \varDelta V_{CE} / \varDelta I_C$ (\textohm)) \\
%            \hline
%            $(...\pm...)\cdot $    & $...\pm...$                                         \\
%            \hline
%        \end{tabular}
%        \caption{Risultati dei fit lineari effettuati sulle misure delle caratteristiche in uscita del transistor BJT,
%            corrispondente a una corrente di base di -200\;\textmu A.}
%        \label{tab:fit-200uA}
%    \end{table}

    I valori ottenuti dai fit sono stati utilizzati per calcolare il valore
    di conduttanza in uscita:

    \begin{equation*}
        \centering
        g = \frac{\varDelta I_C}{\varDelta V_{CE}}
    \end{equation*}

    Si riportano tabulati in seguito i le stime dei valori di conduttanza
    per correnti di base di -100\;\textmu~A e -200\;\textmu~A, calcolati mediante
    i valori riportati rispettivamente nella Tabella~\ref{tab:fit_caratteristiche}.

    \begin{table}[ht]
        \centering
        \begin{tabular}{||c|c||}
            \hline
            \multicolumn{2}{||c||}{Conduttanza di uscita} \\
            \hline
            $I_B$ (\textnormal{\textmu~A}) & $g$ (1/\textohm)) \\
            \hline
            $(-100\pm...) $                & $...\pm...$       \\
            \hline
            $(-200\pm...) $                & $...\pm...$       \\
            \hline
        \end{tabular}
        \caption{stima della conduttanza di uscita per due diversi valori di correnti di base, con incertezze.
        Per approfondire la valutazione delle incertezze si consulti Appendice ...} % TODO: mettere ref appendice
        \label{tab:conduttanza}
    \end{table}

    Utilizzando i valori di conduttanza (Tabella~\ref{tab:conduttanza}) e le misure effettuate per le correnti di
    base ... si vuole calcolare il guadagno di corrente,
    \begin{equation*}
        \centering
        \beta (V_{CE}) = \frac{\varDelta I_C}{\varDelta I_B}
    \end{equation*}

    Al fine di garantire una stima più accurata, sono stati valutati i guadagni
    per diversi valori di tensione $V_{CE}$ e successivamente ne è stata calcolata la media,
    si riporta in seguito il risultato:

    \begin{align*}
        \centering
        \beta = ... \pm ...
    \end{align*}

\end{document}